%-------------------------------------------------------------------------------
%	SECTION TITLE
%-------------------------------------------------------------------------------
\cvsection{Work Experience}


%-------------------------------------------------------------------------------
%	CONTENT
%-------------------------------------------------------------------------------
\begin{cventries}

%%---------------------------------------------------------
  \cvwork
    {Student Researcher (Intern)} % Job title
    {Google} % Organization
    {New York, U.S.A} % Location
    {2025} % Date(s)
    {
      \begin{cvitems} % Description(s) of tasks/responsibilities
        \item{Researched Gemini's video understanding evaluation through auto-generated video-QAs.}
        \item{Develop scalable question auto-generation methods and a correctness verifier for the generated questions.}
      \end{cvitems}
    }

%%---------------------------------------------------------
  \cvwork
    {Graduate Student Researcher} % Job title
    {
        \href{https://sky.cs.berkeley.edu/}{Sky Lab} +
        \href{https://rise.cs.berkeley.edu/}{RISE Lab} +
        \href{https://epic.berkeley.edu/}{EPIC Data Lab}, UC Berkeley
    } % Organization
    {California, U.S.A} % Location
    {2021 - Present} % Date(s)
    {
      \begin{cvitems} % Description(s) of tasks/responsibilities
        \item{Researching efficient video inference systems through spatiotemporal knowledge.
        With cost-based optimization, our system trains proxy ML models to minimize the execution of expensive oracle ML models through input compression and fine-tuning of the oracle models.}
        \item{Designed and developed Spatialyze: a video data analysis system, focusing on geo-spatial-related queries and optimizations. Through our programming paradigm, Spatialyze executes less expensive ML operations by integrating geospatial metadata, achieving more than 2x speed.}
      \end{cvitems}
    }

%%---------------------------------------------------------
  \cvwork
    {Software Engineering Intern} % Job title
    {\href{https://www.forbes.com/sites/janakirammsv/2024/09/30/nvidia-acquires-octoai-to-dominate-enterprise-generative-ai-solutions/}{OctoAI}} % Organization
    {Washington, U.S.A} % Location
    {2020 \& 2021} % Date(s)
    {
      \begin{cvitems} % Description(s) of tasks/responsibilities
          \item{Designed and created a visualizer for deep-learning models and their performance using \underline{\smash{D3+TypeScript}}.}
          \item{The visualizer is a part of the optimizer tool (Octomizer) that optimizes deep-learning models compiled by TVM and measures their performance.}
      \end{cvitems}
    }

%---------------------------------------------------------
  \cvwork
    {Undergraduate + Graduate Research Assistant} % Job title
    {Interactive Data Lab, University of Washington} % Organization
    {Washington, U.S.A} % Location
    {2018 - 2021} % Date(s)
    {
      \begin{cvitems} % Description(s) of tasks/responsibilities
        \item {
          Contributed to \textbf{Vega-Lite} (\href{https://www.github.com/vega/vega-lite}{\textit{github.com/vega/vega-lite}}), a \underline{web-based} high-level grammar of interactive graphic for generating easy-to-understand visualization.
          Designed and implemented grammar for creating error bars/error bands,
          enabling users to create error bar/band charts without the need to manually compose different types of marks.
          As a result, the specification for creating an error bar chart is \underline{\smash{shortened by half}}.
        }
        \item {
          Contributed to \textbf{Arquero} (\href{https://www.github.com/chanwutk/arquero-sql}{\textit{github.com/chanwutk/arquero-sql}}), a query processing library in JS.
          Designed and implemented Arquero-SQL as an alternative execution engine to Arquero's own engine in JS.
          Arquero-SQL executes Arquero queries on an SQL server to achieve better performance and scalability.
        }
      \end{cvitems}
    }

%---------------------------------------------------------
  % \cvwork
  %   {Graduate Teaching Assistant} % Job title
  %   {Paul G. Allen Center for Computer Sci. \& Eng., University of Washington} % Organization
  %   {Washington, U.S.A} % Location
  %   {2020 - 2021} % Date(s)
  %   {
  %     \begin{cvitems} % Description(s) of tasks/responsibilities
  %       \item {
  %         Courses: \textbf{1.} Software Design \& Implementation \textbf{2.} Data Visualization
  %       }
  %       \item {
  %         Duties: Hold office hours for students, grade student's assignments, and teach students in teaching assistant section.
  %       }
  %       \item {
  %         Teach supplementary materials to the courses' lecture in sections with small group of students. Review complicated concepts such as D3 Tutorial.
  %       }
  %     \end{cvitems}
  %   }

%%---------------------------------------------------------
  \cvwork
    {Software Engineering Intern (Product Development: Manage \& Optimize)} % Job title
    {\href{https://www.docusign.com/}{DocuSign}} % Organization
    {Washington, U.S.A} % Location
    {2019} % Date(s)
    {
      \begin{cvitems} % Description(s) of tasks/responsibilities
        \item {
          Design \& develop AWS services as part of the Advanced Analytics Platform, used to extract, sanitize, and store usage data.
        }
        \item {
          Design AWS Lambdas to automatically shut down idle EMR clusters and notebooks, preventing unnecessary billing.
          Design spark jobs to keep the schema of the service's database up-to-date and to clean up unused files from failed data ingestion, thereby preventing dirty data from being analyzed.
        }
      \end{cvitems}
    }

%---------------------------------------------------------
\end{cventries}
