%-------------------------------------------------------------------------------
%	SECTION TITLE
%-------------------------------------------------------------------------------
\cvsection{Projects}


%-------------------------------------------------------------------------------
%	CONTENT
%-------------------------------------------------------------------------------
\begin{cventries}

%---------------------------------------------------------
  \cventry
    {\href{http://www.github.com/chanwutk/pleiades}{github.com/chanwutk/pleiades}} % Role
    {Pleiades: Interactive Composing Tools for Vega-Lite Charts} % Event
    {Washington, U.S.A} % Location
    {April 2019 - June 2019} % Date(s)
    {
      \begin{cvitems} % Description(s)
        \item {
          A web-based software created using React and TypeScript that provides a graphical user interface for users to compose and layout Vega-Lite charts.
          Instead of configuring with Vega-Lite JSON specification to layout and compose multiple charts, user can easily interact with the graphical user interface of Pleiades to compose, layout, and edit them.
        }
        \item {
          The project has been developed in collaboration with Manesh Jhawar and Sorawee Porncharoenwase.
          The paper discussing the product can be found here
          \href{https://chanwutk.github.io/pleiades/paper.pdf}{
            \textit{chanwutk.github.io/pleiades/paper.pdf}
          }.
          And, the application can be found here
          \href{https://chanwutk.github.io/pleiades/app.html}{
            \textit{chanwutk.github.io/pleiades/app.html}
          }.
        }
      \end{cvitems}
    }

  \cventry
    {\href{http://github.com/chanwutk/pokemon-go-forecast}{github.com/chanwutk/pokemon-go-forecast}} % Role
    {Pekémon Go Weather forecast application} % Event
    {Bangkok, Thailand} % Location
    {September 2018 - Present} % Date(s)
    {
      \begin{cvitems} % Description(s)
        \item {
          Web application to predict the weather for Pokémon Go;
          players can plan to catch Pokémons that get benefit from the weather predicted.
        }
        \item {
          Created API server using TypeScript and Node.js to retrieve and store weather information from AccuWeather API.
        }
        \item {
          Created front website to visualize weather information from the API server.
        }
      \end{cvitems}
    }

    \cventry
    {\href{http://github.com/chanwutk/one-handed-braille-keyboard}{github.com/chanwutk/one-handed-braille-keyboard}} % Role
    {One Handed Braille Keyboard at DubHacks 2019} % Event
    {Washington, U.S.A} % Location
    {October 2019 - October 2019} % Date(s)
    {
      \begin{cvitems} % Description(s)
        \item {
          The current touchscreen keyboard layout for visually impaired users requires two hands to operates and requires users to change hands position.
        }
        \item {
          We would like to provide an alternative for users to use their touchscreen keyboard with only one hand so that users can multitask between typing and other thing else.
        }
        \item {
          Our keyboard has 4 large buttons as we want the users to be able to type without looking.
          One braille alphabet is created by a combination of 3 keystrokes, which then create an English alphabet.
        }
      \end{cvitems}
    }

%---------------------------------------------------------
\end{cventries}
